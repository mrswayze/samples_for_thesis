% ~~~~ PAKETE ~~~~ %

\usepackage[automark]{scrlayer-scrpage} %brauchen wir für die Optik
\usepackage{euler}	% Benutzen die AMS euler Schriftart für den Mathemodus
\usepackage{fontspec} %zeigt an, dass XeLateX verwendet wird
\usepackage[T1]{fontenc}
\usepackage[onehalfspacing]{setspace} %Setzt Zeilenabstand auf 1.5fache des üblichen
\usepackage[ngerman]{babel} %deutsche Sprache woop woop

% Mathematische Symbole

\usepackage{amssymb, amsmath, amsbsy} 
\usepackage{bm, bbm}
\usepackage[ntheorem]{mathtools}
\usepackage{complexity} % Befehle für Komplexität: \NP, ...
\usepackage{marvosym}   
%\usepackage[scr]{rsfso} % \mathscr{} sieht dann hübscher aus
%\usepackage[retainorgcmds]{IEEEtrantools} % Bietet sich an, wenn man viel mit Gleichungen arbeitet


% Bildunterschriften 

\usepackage[textfont={sl},font={small},%
  labelsep=space, textformat=period]{caption}
\usepackage{subcaption}


% Hyperlinks im Dokument

\usepackage[]{hyperref}
 \hypersetup{colorlinks=true, citecolor=mLightBrown,%
                 filecolor=black,%
                 linkcolor=mDarkTeal,%
                 urlcolor=black}

               

% Umgebungen für Sätze             
\usepackage[amsmath, thref, thmmarks, hyperref]{ntheorem} %mathematische Sprachen einbinden


% Einbinden von Grafiken
\usepackage{graphicx} %Graphiken


% Automatisch Deutsches Datum
\usepackage[ngerman]{datetime}
\newdateformat{myformat}{\THEDAY{ter }\monthnamengerman[\THEMONTH], \THEYEAR}

% Falls Symbolverzeichnis geünwscht:

\usepackage[symbols, automake, stylemods={topic}, postpunc={comma}]{glossaries-extra}
%sort=none, nogroupskip,
\usepackage{glossary-mcols}
\usepackage[bb=boondox]{mathalpha}



% ~~~~~~ ZUSATZEINSTELLUNGEN ~~~~~

% Schriftart: CMU Serif
\setmainfont{CMU Serif}


%Figure Captions
\renewcommand{\figurename}{\textsc{Fig.}}


% Gleichungen nach Kapiteln nummerieren
\numberwithin{equation}{chapter}

% ~~~~~ THEOREME usw. ~~~

% Style für Sätze
\makeatletter
\newtheoremstyle{restated}
  {\item[\hskip\labelsep \theorem@headerfont ##1\ ##2\theorem@separator]}%
  {\item[\hskip\labelsep {\theorem@headerfont ##1\ ##2}{\normalfont\ (##3)}{\theorem@headerfont
  \theorem@separator}]}
\makeatother

\makeatletter
\newtheoremstyle{nonumberrestated}%
  {\item[\theorem@headerfont \hskip\labelsep ##1\theorem@separator]}%
  {\item[\hskip\labelsep \theorem@headerfont ##3\theorem@separator]}%
\makeatother
  

% Sätze
\theoremstyle{restated}
\theoremnumbering{arabic}
\theoremheaderfont{\normalfont\bfseries}
\theorembodyfont{\slshape}
\theoremseparator{.}
\theorempreskip{2ex}
\theorempostskip{\topsep}
\theoremindent0cm
\theoremnumbering{arabic}
\theoremsymbol{\ensuremath{_\diamond}} % kleine Raute am Ende jedes Theorems, wenn nicht gewünscht, einfach weg damit


\newtheorem{theorem}{Theorem}[chapter] % Sätze werden nach Kapiteln nummeriert
\newtheorem{theoremstar}{Theorem} 	   % Metaresultate, die unabhängig vom Kapitel nummeriert werden sollen


% Sonstige Umgebungen

\theoremstyle{restated} 
\newtheorem{lemma}[theorem]{Lemma}
\newtheorem{proposition}[theorem]{Proposition}
\newtheorem{corollary}[theorem]{Corollary}
\newtheorem{conjecture}[theorem]{Conjecture}
\newtheorem{observation}[theorem]{Observation}
\newtheorem{question}[theorem]{Question}
\newtheorem{problem}[theorem]{Problem}
\newtheorem{claim}{Claim}[theorem]


%Falls Theoreme in der Arbeit mehrfach verwendet werden

\makeatletter

\def\ifempty#1{%
 \def\@tmp@a{#1}%
 \ifx\@tmp@a\@empty%
}

\theoremstyle{restated}
\newtheorem*{restatedtheorem*}{}

\newenvironment{restatement}[2][]{%
 \ifempty{#1}
  \begin{restatedtheorem*}[\autoref*{#2}]%
 \else%
  \begin{restatedtheorem*}[\autoref*{#2}{\normalfont\ (#1)}]%
 \fi%
}%
{\end{restatedtheorem*}}
\makeatother



% Umgebung für Beweise

\makeatletter
\newtheoremstyle{proof}%
{\item[\hskip\labelsep \theorem@headerfont ##1\theorem@separator]}%
{\item[\hskip\labelsep \theorem@headerfont ##1\ ##3\theorem@separator]}
\makeatother

\theoremstyle{proof}
\theoremheaderfont{\normalsize\normalfont\itshape}
\theorempreskip{.5ex}
\theorempostskip{\topsep}
\theoremseparator{.}

\theoremsymbol{\ensuremath{_\blacksquare}} %schwarzes Kästchen als Beweissymbol

\theorembodyfont{\upshape}
\newtheorem{proof}{Proof}

\theorempreskip{0ex}

\theoremsymbol{\ensuremath{_\square}}% weißes Kästchen als Beweissymbol (z.b. für Claims innerhalb von Beweisen)
\newtheorem{proof2}{Proof} % \begin{proof2}... gibt dann Beweis mit weißem Kästchen


%Umgebungen "Proof of" und "Alternate proof of" falls Beweise erst zu einem späteren Zeitpunkt geführt werden

\theorembodyfont{\upshape}
\theoremseparator{.}
\theorempreskip{3.5ex}
\theorempostskip{\topsep}
\theoremsymbol{\ensuremath{_\blacksquare}}
\theoremheaderfont{\large\normalfont\itshape}
\newtheorem{proofof}{Proof of}
\newtheorem{proofalternate}{Alternate proof of}


%~~~~ HILFSBEFEHLE ~~~

% In Latex lassen sich einfach neue Befehle definieren:
% Syntax: \newcommand{\neuername}[Anzahl Eingabeparameter] {Das hier wird ausgeführt, #1 greift auf Parameter 1 zu usw}

\newcommand{\set}[2]{\{#1 \;|\; #2 \}} 
\newcommand{\case}[2]{\indent {\bfseries Case #1\/:} {\itshape #2.}}

%
%%DP-Aussehen
%\renewcommand{\DP}{\textsf{\upshape{DP}}}


%Pfeile für gerichtete Graphen
\newcommand{\bd}[1]{\overset{\text{\tiny$\bm\leftrightarrow$}}{#1}}
\newcommand{\overr}[1]{\overset{\text{\tiny$\bm\rightarrow$}}{#1}}
\newcommand{\overl}[1]{\overset{\text{\tiny$\bm\leftarrow$}}{#1}}


%Kanten in Kritischen Di-Graphen
\newcommand{\dext}{\overr{\text{\upshape ext}}}
\newcommand{\Dext}{\overr{\text{\upshape Ext}}}
\newcommand{\ext}{\text{\upshape{ext}}}
\newcommand{\Ext}{\text{\upshape{Ext}}}


% ~~~~ TIKZ ~~~~

% Für Abschlussarbeiten in der Graphentheorie bietet es sich an, Bilder mit TIKZ zu zeichnen
% TIKZ DOKU: https://texdoc.org/serve/tikz/0

\usepackage{tikz}

%Für TIKZ gibt es diverse nützliche Libraries:

\usetikzlibrary[graphs, arrows, backgrounds, intersections, positioning, fit, petri, calc, shapes, decorations.pathmorphing, shapes.geometric, mindmap, patterns, arrows.meta]  

% Default-Werte:

\tikzset{node distance=1cm, bend angle=20,
vertex/.style={circle,minimum size=2mm,very thick, draw=black, fill=black, inner sep=0mm},
fvertex/.style={circle,minimum size=5mm,  draw=black, inner sep=0mm},
help lines/.style={-,color=black, >=stealth', shorten <=.5pt, shorten >=.5pt}}


%Farben für TIKZPICTURES
\definecolor{mDarkBrown}{HTML}{604c38}
\definecolor{mDarkTeal}{HTML}{0E668B}
\definecolor{mLightBrown}{HTML}{EB811B}
\definecolor{mLightGreen}{HTML}{14B03D}
\definecolor{mLightBlue}{HTML}{88CCEE}
\definecolor{TUgreen}{RGB}{0,102,102}


\definecolor{TolDarkPurple}{HTML}{332288}
\definecolor{TolLightBlue}{HTML}{88CCEE}
\definecolor{TolLightGreen}{HTML}{44AA99}
\definecolor{TolDarkGreen}{HTML}{117733}
\definecolor{TolDarkBrown}{HTML}{999933}
\definecolor{TolLightBrown}{HTML}{DDCC77}
\definecolor{TolDarkRed}{HTML}{661100}
\definecolor{TolLightRed}{HTML}{CC6677}
\definecolor{TolLightPink}{HTML}{AA4466}
\definecolor{TolDarkPink}{HTML}{882255}
\definecolor{TolLightPurple}{HTML}{AA4499}



% ~~~~~~~~ BIBLATEX: Einfaches Einbinden von Literatur ~~~~~~~~~~
% WICHITG: Als Befehl zum Bib(la)tex-Kompilieren muss biber % angegeben werden 
% BIBLATEX DOKU: http://mirrors.ibiblio.org/CTAN/info/translations/biblatex/de/biblatex-de-Benutzerhandbuch.pdf

\usepackage[
backend=biber, 	% Benutzt zum Übersetzen biber anstatt bibtex
sorting=nyvt, 	% Sortieren der Einträge: name > year > volume > title 
%				% Für andere Möglichkeiten siehe https://tex.stackexchange.com/questions/51434/biblatex-citation-order  
hyperref=true,	% Erzeugt Hyperlinks ins Literaturverzeichnis
style=numeric,	% Einträge haben die Form [1], [2], etc
url=true,		% Falls "url" option im bib-file genutzt wird, fügt dies automatisch Hyperlink ein
block=none,		% Keine Extra-Abstände zwischen den Einträgen
eprint=true,	% Elektronische Kennzeichnung einer Online Publikation
natbib=true,	% Dadurch kann das Zusatzpaket natbib genutzt werden
maxnames=100,	% damit im Literaturverzeichnis keine Autoren abgeschnitten werden
backref = true]	% Gibt im Literaturverzeichnis an, auf welcher Seite der jeweilige Eintrag zitiert wird
{biblatex}

\renewcommand*{\mkbibnamefamily}[1]{\textsc{#1}} 	% Autoren in Kapitälchen
\renewcommand*{\labelnamepunct}{:\addspace}			% Doppelpunkt nach den Autoren
\renewbibmacro{in:}{}								% Sonst steht in: vor jedem Journal, das braucht ja wohl niemand
\DeclareNameAlias{default}{family-given}				% Erst Nachname, dann Vorname
\DeclareFieldFormat[article]{volume}{\textbf{#1}}	% Volume in fett bei @article
\DeclareFieldFormat{pages}{#1}						% Seitenzahlen ohne alles
\DeclareFieldFormat[article]{title}{{#1}}			% Titel ohne alles bei @article
\DeclareFieldFormat[unpublished]{title}{{#1}}		% Gleiches für @unpublished
\DeclareFieldFormat[thesis]{title}{{#1}}			% und für @thesis
\DeclareFieldFormat[online]{title}{{#1}}			% ...
\DeclareFieldFormat[book]{title}{{#1}}				% ... ⊙﹏⊙
\DeclareFieldFormat[inproceedings]{title}{{#1}}		% geschafft
\DeclareFieldFormat[inproceedings]{booktitle}{\upshape{#1}} 
\DeclareFieldFormat[incollection]{title}{{#1}}
\DeclareFieldFormat[incollection]{booktitle}{In: \emph{#1}}
\finalandcomma 	% Komma vor dem 'and' bei mehreren Autoren



% ~~~~~~ VORWORT (optional) ~~~~~~~~~~~
       
%% Falls eine nicht-nummerierte Einleitung gewünscht ist, kann diese durch \foreword[Lauftitel]{richtiger Titel} erzeugt werden
\DeclareNewSectionCommand[%
  style=chapter,
  beforeskip= -3\baselineskip minus \parskip,
  afterskip= 1.5\baselineskip plus .1\baselineskip minus .167\baselineskip,
  innerskip=0.5\baselineskip,
  level=0,
  pagestyle=plain,
  tocindent=0pt,
  tocstyle=chapter,
  tocnumwidth=0pt,
]{foreword}
\makeatletter


\renewcommand{\forewordformat}{\centering \bfseries 
 \Huge Einleitung\\
    \noindent\rule{\linewidth}{0.2pt}\par 
 }
 
\renewcommand*{\forewordmarkformat}{Einleitung:}

\newcommand*{\theforeword}{}




%~~~~~~ KOMA-Script ~~~~~~~~%
% KOMA-Script ist eine ziemlich komplexe Sammlung von Klassen und Paketen zum typesetting
% Die KOMA-Doku findet sich hier: https://komascript.de/~mkohm/scrguide.pdf
% An den nächsten Zeilen am Besten nicht herumpfuschen


\KOMAoptions{headsepline=true, headings=twolinechapter, headinclude=true, footinclude=false, bibliography=totoc, titlepage=true, cleardoublepage=plain}
\pagestyle{scrheadings}
\clearpairofpagestyles 
\ihead[]{}
\ohead{\pagemark}
\chead{\headmark}
\cfoot[]{}
\renewcommand{\chapterformat}{\centering \bfseries 
 \Huge \chapapp \mbox{} \thechapter \\
    \noindent\rule{\linewidth}{0.2pt}\par 
 }
\renewcommand{\chaptermarkformat}{\thechapter \mbox{ }}
\newcommand*{\ORIGchapterheadendvskip}{}% 
\let\ORIGchapterheadendvskip=\chapterheadendvskip 
\renewcommand*{\chapterheadendvskip}{% 
  {% 
    \setlength{\parskip}{0pt}% 
    \vspace{0.5 cm}
    \noindent\rule{\linewidth}{0.2pt}\par 
  }% 
  \ORIGchapterheadendvskip 
} 
\renewcommand{\raggedchapter}{\raggedpart }
\renewcommand*{\chaptermarkformat}{} 



\renewcommand{\partformat}{\centering \bfseries 
 \Huge Teil~\thepart \\
 \vspace{1.5cm}
    \noindent\rule{\linewidth}{0.4pt}\par 
 }
\renewcommand{\partmarkformat}{\thepart \mbox{ }}
\newcommand*{\ORIGpartheadendvskip}{}% 
\let\ORIGpartheadendvskip=\partheadendvskip 
\renewcommand*{\partheadendvskip}{% 
  {% 
    \setlength{\parskip}{0pt}% 
    \vspace{0.5 cm}
    \noindent\rule{\linewidth}{0.4pt}\par 
  }% 
  \ORIGpartheadendvskip 
} 

%\setkomafont{chapterprefix}
\addtokomafont{disposition}{%\normalfont 
\bfseries}

\addtokomafont{part}{\Huge \scshape %\normalfont
\mdseries
}

\addtokomafont{chapter}{\huge \scshape %\normalfont
\mdseries
}
\addtokomafont{section}{\large}
\addtokomafont{minisec}{%\large
\centering}

 




% ~~~~~ BEI SEHR VIEL LANGEWEILE ~~~~~
%\Komaoptions{Option=Wert} oder \Komaoption{Option}{Wert}

%Alle möglichen Komaoptionen können hier eingegeben werden
%mögliche Optionen: 
%--------------
%Satzspiegel:
%BCOR=Korrektur ~> absoluter Wert der Bindekorrektur; kann auch bei \documentclass direkt eingegeben werden
%DIV=Faktor ~> Je höher, umso kleiner werden die Ränder
%DIV=calc ~> typearea berechnet in Abhängigkeit von gewählter Schriftart einen guten Satzspiegel
%DIV=last ~> Satzspiegelberechnung mit denselben DIV-Argumenten erneut durchführen, 
%			 alternativ: \recalctypearea
%headinclude=on/off, footinclude=on/off ~> Zählen Kopf-/Fußzeile zum Textkörper?
%headlines=Zeilenanzahl ~> Anzahl der Kopfzeilen im Allgemeinen, voreingestellt:1,25; 
%footlines=Zeilenanzahl ~> analog
%danach: unbedingt Satzspiegel neu berechnen!
%--------------
%Dokumenttitel:
%titlepage=on/off
%abstract=on/off ~> Überschrift über Zusammenfassung ein/ausgeschaltet
%--------------
%Inhaltsverzeichnis:
%toc=Einstellung
%--------------
%Absatzauszeichnung:
%parskip=Methode, wobei Methode=full/half */+/-, full ~> Absatzabstand halbe/ganze Zeile, Symbol:Abstand %am Zeilenende
%wahrscheilich besser: Absatzeinzug, ist voreingestellt
%--------------
%Kopf-/Fußzeile:
%headsepline=true/false, footsepline=true/false ~> Linie zischen Kopf-/Fußzeile und Text
%ist die Option true, so zählt Kopf-/Fußzeile automatisch zum Textkörper ~> \recalctypearea verwenden
%cleardoublepage=Seitenstil ~> Stil von Vakatseiten
%footnotes=multiple ~> aufeinanderfolgende Fußnoten werden mit Kommata getrennt
%--------------
%Gliederung:
%open=right ~> neue Kapitel beginnen auf der rechten Seite
%chapterprefix=true/false ~> Kapitel 1...
%appendixprefix=true/false ~> ANHANG
%headings=small/normal/big ~> Überschriftengröße
%headings=optiontohead/optiontotoc~> [] bei Kapitel kommt automatisch als Kolumnentitel/in den TOC
%numbers=endperiod/noendperiod ~> Punkt/keinPunkt nach Elementen im TOC
%---------------------------------------------------------------------------------------------------%

%Titelseite:
%\titlepagestyle{Wert}
%\titlehead{Kopf} ~> Kopf der Titelseite, Gestaltung über \setkomafont{titlehead}{...}
%\subject{Typisierung} ~> wird unmittelbar über dem Titel ausgegeben
%\title{Titel}
%\subtitle{Subtitel}
%\author{Autor}
%\date{Datum}
%\publishers{Verlag} ~> hier der Betreuer?
%\and ~> falls mehrere Autoren
%\dedication{Widmung} ~> kommt auf neue Seite
%\maketitle ~> Titelseite erzeugen
%\begin{abstract}....\end{abstract} ~> Erzeugt Umgebung für Zusammenfassung der Arbeit
%
%Inhaltsverzeichnis:
%Schrift der oberen Einträge: \setkomafont{chapterentry}{...}
%Schrift der Seitenzahlen: chapterentrypagenumber
%Schrift der punktierten Verbindungslinien: chapterentrydots
%
%Textauszeichungen:
%\textsuperscript{Text}, \textsubscript{Text} ~> Text hoch-/tiefstellen
%\minisec{Text} ~> Miniüberschrift
%
%Seitenstil:
%\pagestyle{Seitenstil} 
%\thispagestyle{lokaler Seitenstil} 
%\chapterpagestyle{Seitenstil bei Kapitelanfängen}
%\indexpagestyle. Es existieren folgende Möglichkeiten:
%~> empty: Kopf- und Fußzeile komplett leer
%~> headings: Überschriften werden automatisch in Seitenkopf übernommen
%~> myheadings: wie headings, aber mehr Freiheiten ~>\markboth{linkeMarke}{rechteMarke}
%~> plain: nur Seitenzahl wird ausgegeben
%
%Kopf-/Fußzeile:
%\footnote[Nummer]{Text}
%\footref{Referenz} ~> setzt man nach einer Fußnote ein label, so kann man sich durch footref darauf %beziehen
%\deffootnote[Markenbreite]{Einzug}{Absatzeinzug}{Markendefinition} ~> Absatzeinzug für Absätze in FN
%\deffootnotemark{...} ~> für Fußnotenmarkierung im Text
%\setfootnoterule[Höhe]{Länge}~> genaue Werte für Linie über Fußnoten festlegen
%\setkomafont{footnoterule}{....} ~> zum Beispiel Farbe verändern
% 
%\pagenumbering{}: arabic, roman, alph; Seitenzahl wird beim Aufruf immer zurückgesetzt
%\setkomafont{Element}{Befehle} ~> Schriftarten des jew. Elements verändern
%\addtokomafont{Element}{Befehle}
%\usekomafont{Element}{Befehle} 
%
%\clearpage, \cleardoublepage: alle noch nicht ausgegebenen Gleitumgebungen ausgeben, neue Seite beginnen
%\cleardoubleoddstandardpage: vor neuem Kapitel verwenden
%
%
%Umgebungen:
%\begin{itemize}....\end{itemize} ~> Liste
%\begin{enumerate}...\end{enumerate} ~> nummerierte Liste
%\begin{description}...\end{description} ~> Stichworliste; Beschreibung einzelner Begriffe
%
%\begin{labeling}[Trennzeichen]{Längstes Muster}
%\item[Muster]
%\end{labeling} ~> Abstand zwischen Begriff und Erklärung festsetzbar; setkomafont{labelinglabel}{..}
%
%\begin{quote}...\end{quote} ~> Anekdote, Absätze werden durch Abstände gekennzeichnet
%\begin{quotation}...\end{quotation} ~> Absätze werden eingerückt